%----------------------------------------------------------------------------
\chapter*{Köszönetnyilvánítás}
\addcontentsline{toc}{chapter}{Köszönetnyilvánítás}
%----------------------------------------------------------------------------

Köszönetet mondok családomnak, hogy szeretettel viselték, amikor esténként munkából hazatérve ahelyett, hogy velük együtt -- családi körben -- töltöttem volna a vacsorát, szégyentelenül néhány diétás sonkás sajtos szalámis szendviccsel és egy liter tejjel elvonultam és számítógépem társaságában görnyedtem végig az estét, majd az éjszakát.

Köszönöm anyai nagymamámnak, hogy jelentős mennyiségű konverzibilis tárgyi eszközt bocsájtott rendelkezésemre -- az általa biztosított cukros kekszgolyók ATP-re konvertálása még a legsötétebb éjjeli órák idején is képes volt újabb (a felhasználások számával lineárisan növekvő konstanssal amortizált) 10 perc munkára elegendő energiát biztosítani.

Köszönöm apai nagymamámnak, hogy házi baracklekvárával és paradicsomlevével felvidította a reggeli ébredés pillanatait -- ekkor mindig feledni tudtam a hátalévő fejezetek számát. Hálás vagyok neki, hogy szívén viseli szakbarbarizmusom előrehaladott állapotát és mindig továbbküldi a kulturális témájú zenés prezentációkat tartalmazó körleveleket -- bennem a fentebb stíl művelését elősegítendő.

Köszönöm apai nagyapámnak, hogy már akkor bevezetett a tranzisztorok varázslatos világába, amikor még az Ohm törvényben található betűket sem ismertem -- ha nincs az ő szelíd, de határozott gondoskodása, amivel az egyetlen igaz életpálya felé terelt, most talán egy vidéki színjátszó szakkör felolvasópróbáján ülnék vagy egy székmentes keleti jellegű pesti teázó padlóján próbálnék egy alárendelő módon többszörösen bővített, didaktikus elemekkel gondosan körülvett mondatot még kacifántosabbá tenni.

Köszönöm anyai nagyapámnak, hogy megmutatta, hogyan kell nyulat nyúzni, kulcsot másolni és hogy milyen szép a kilátás a kilencemeletes tetejéről. Biztos vagyok benne, hogy születésnapi köszöntései nagymértékben inspiráltak.

Köszönöm dédszüleimnek -- különösen anyai-apai dédanyámnak -- hogy folyamatos lelki támogatást nyújtottak a legelkeseredettebb időkben is. Dédi! Tudom, hogy megígértem, hogy idős korodban majd gondodat viselem. Nem felejtettelek el, mégha ritkán beszélünk is! Még van egy-két elintézni valóm, de hamarosan találkozunk.

Köszönöm barátaimnak, hogy szeretnek és elfogadnak, valamint biztosítanak munkám nélkülözhetetlenségéről még akkor is, amikor a dolgozatom címének részletezése közben tekintetük kissé ködössé válik. Külön köszönöm, hogy egészségemmel nem törődve születésnapom alkalmából megajándékoztak 5 teljes rúd $\int_{-\infty}^\infty f(x)\ e^{- 2\pi i x \xi}\,dx$ típusú csokis keksszel. Ezek majdnem végig kitartottak, csupán a C++ szabvány olvasása közben meginduló kegyetlen ostromnak nem tudtak ellenállni. Nélkületek nem sikerült volna.

Köszönöm konzulensemnek, Rajacsics Tamásnak, hogy a teljes írást a kézhezvétel után szinte azonnal végigolvasta és azt értékes észrevételeivel gazdagította.

Köszönöm a Morgan Stanley Magyarország Elemző Kft-nek, hogy finanszírozta a dolgozat alapját adó program elkészítését és megengedte a fejlesztés során összegyűjtött tapasztalatok és kutatási eredmények publikálását. Szívet melengető érzés, hogy három manager is elolvasta a releváns fejezeteket, tőlük ezúton kérek elnézést, én magamtól soha, (soha!) nem írnék ennyit.

A dolgozat nem jöhetett volna létre számos nagyszerű -- többnyire nyílt forráskódú -- program nélkül. Ezek közül a teljesség igénye nélkül néhány: Linux, GCC, Git, \LaTeX, Gnome, vim, Firefox. Köszönet a fejlesztőknek és tamogatóknak!

\vfill
\begin{center}
\emph{Mondottam, ember, küzdj és bízva bízzál!}
\end{center}
\thispagestyle{empty}
